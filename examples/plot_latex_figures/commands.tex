% horizontal rule equivalent to \cline but with 2pt width
\newcommand{\Cline}[1]{%
 \noalign{\global\setlength\Origarrayrulewidth{\arrayrulewidth}}%
 \noalign{\global\setlength\arrayrulewidth{2pt}}\cline{#1}%
 \noalign{\global\setlength\arrayrulewidth{\Origarrayrulewidth}}%
}

% draw a vertical rule of width 2pt on both sides of a cell
\newcommand\Thickvrule[1]{%
  \multicolumn{1}{!{\vrule width 2pt}c!{\vrule width 2pt}}{#1}%
}

% draw a vertical rule of width 2pt on the left side of a cell
\newcommand\Thickvrulel[1]{%
  \multicolumn{1}{!{\vrule width 2pt}c|}{#1}%
}

% draw a vertical rule of width 2pt on the right side of a cell
\newcommand\Thickvruler[1]{%
  \multicolumn{1}{|c!{\vrule width 2pt}}{#1}%
}

%\theoremstyle{plain}
%\newtheorem{theorem}{Satz}
%\newtheorem{lemma}[theorem]{Lemma}
%\newtheorem{proposition}[theorem]{Proposition}
%\newtheorem{Theorem}[theorem]{Theorem}
%\newtheorem{corollary}[theorem]{Korollar}
%
%\theoremstyle{definition}
%\newtheorem{definition}[theorem]{Definition}
%\newtheorem{algorithmThm}[theorem]{Algorithmus}
%\newtheorem{remark}[theorem]{Bemerkung}
%\newtheorem{example}[theorem]{Beispiel}
%\newtheorem{outlook}[theorem]{Ausblick}
%\newtheorem{notation}[theorem]{Notation}
%
%%\numberwithin{equation}{section}
%\numberwithin{theorem}{section}

\newcommand{\TODO}[1]{\textcolor{red}{[TODO: #1]}}
\newcommand{\DaH}[1]{\textcolor{blue}{[DH: #1]}}
\newcommand{\FF}[1]{\textcolor{red}{[FF: #1]}}
\newcommand{\DP}[1]{\textcolor{blue}{[DP: #1]}}


\newcommand{\bbA}{\mathbb{A}}
\newcommand{\bbB}{\mathbb{B}}
\newcommand{\bbC}{\mathbb{C}}
\newcommand{\bbD}{\mathbb{D}}
\newcommand{\bbE}{\mathbb{E}}
\newcommand{\bbF}{\mathbb{F}}
\newcommand{\bbG}{\mathbb{G}}
\newcommand{\bbH}{\mathbb{H}}
\newcommand{\bbI}{\mathbb{I}}
\newcommand{\bbJ}{\mathbb{J}}
\newcommand{\bbK}{\mathbb{K}}
\newcommand{\bbL}{\mathbb{L}}
\newcommand{\bbM}{\mathbb{M}}
\newcommand{\bbN}{\mathbb{N}}
\newcommand{\bbO}{\mathbb{O}}
\newcommand{\bbP}{\mathbb{P}}
\newcommand{\bbQ}{\mathbb{Q}}
\newcommand{\bbR}{\mathbb{R}}
\newcommand{\bbS}{\mathbb{S}}
\newcommand{\bbT}{\mathbb{T}}
\newcommand{\bbU}{\mathbb{U}}
\newcommand{\bbV}{\mathbb{V}}
\newcommand{\bbW}{\mathbb{W}}
\newcommand{\bbX}{\mathbb{X}}
\newcommand{\bbY}{\mathbb{Y}}
\newcommand{\bbZ}{\mathbb{Z}}

\newcommand{\calA}{\mathcal{A}}
\newcommand{\calB}{\mathcal{B}}
\newcommand{\calC}{\mathcal{C}}
\newcommand{\calD}{\mathcal{D}}
\newcommand{\calE}{\mathcal{E}}
\newcommand{\calF}{\mathcal{F}}
\newcommand{\calG}{\mathcal{G}}
\newcommand{\calH}{\mathcal{H}}
\newcommand{\calI}{\mathcal{I}}
\newcommand{\calJ}{\mathcal{J}}
\newcommand{\calK}{\mathcal{K}}
\newcommand{\calL}{\mathcal{L}}
\newcommand{\calM}{\mathcal{M}}
\newcommand{\calN}{\mathcal{N}}
\newcommand{\calO}{\mathcal{O}}
\newcommand{\calP}{\mathcal{P}}
\newcommand{\calQ}{\mathcal{Q}}
\newcommand{\calR}{\mathcal{R}}
\newcommand{\calS}{\mathcal{S}}
\newcommand{\calT}{\mathcal{T}}
\newcommand{\calU}{\mathcal{U}}
\newcommand{\calV}{\mathcal{V}}
\newcommand{\calW}{\mathcal{W}}
\newcommand{\calX}{\mathcal{X}}
\newcommand{\calY}{\mathcal{Y}}
\newcommand{\calZ}{\mathcal{Z}}

\providecommand{\bfa}{\boldsymbol{a}}
\providecommand{\bfb}{\boldsymbol{b}}
\providecommand{\bfc}{\boldsymbol{c}}
\providecommand{\bfd}{\boldsymbol{d}}
\providecommand{\bfe}{\boldsymbol{e}}
\providecommand{\bff}{\boldsymbol{f}}
\providecommand{\bfg}{\boldsymbol{g}}
\providecommand{\bfh}{\boldsymbol{h}}
\providecommand{\bfi}{\boldsymbol{i}}
\providecommand{\bfj}{\boldsymbol{j}}
\providecommand{\bfk}{\boldsymbol{k}}
\providecommand{\bfl}{\boldsymbol{l}}
\providecommand{\bfm}{\boldsymbol{m}}
\providecommand{\bfn}{\boldsymbol{n}}
\providecommand{\bfo}{\boldsymbol{o}}
\providecommand{\bfp}{\boldsymbol{p}}
\providecommand{\bfq}{\boldsymbol{q}}
\providecommand{\bfr}{\boldsymbol{r}}
\providecommand{\bfs}{\boldsymbol{s}}
\providecommand{\bft}{\boldsymbol{t}}
\providecommand{\bfu}{\boldsymbol{u}}
\providecommand{\bfv}{\boldsymbol{v}}
\providecommand{\bfw}{\boldsymbol{w}}
\providecommand{\bfx}{\boldsymbol{x}}
\providecommand{\bfy}{\boldsymbol{y}}
\providecommand{\bfz}{\boldsymbol{z}}

\providecommand{\bfA}{\boldsymbol{A}}
\providecommand{\bfB}{\boldsymbol{B}}
\providecommand{\bfC}{\boldsymbol{C}}
\providecommand{\bfD}{\boldsymbol{D}}
\providecommand{\bfE}{\boldsymbol{E}}
\providecommand{\bfF}{\boldsymbol{F}}
\providecommand{\bfG}{\boldsymbol{G}}
\providecommand{\bfH}{\boldsymbol{H}}
\providecommand{\bfI}{\boldsymbol{I}}
\providecommand{\bfJ}{\boldsymbol{J}}
\providecommand{\bfK}{\boldsymbol{K}}
\providecommand{\bfL}{\boldsymbol{L}}
\providecommand{\bfM}{\boldsymbol{M}}
\providecommand{\bfN}{\boldsymbol{N}}
\providecommand{\bfO}{\boldsymbol{O}}
\providecommand{\bfP}{\boldsymbol{P}}
\providecommand{\bfQ}{\boldsymbol{Q}}
\providecommand{\bfR}{\boldsymbol{R}}
\providecommand{\bfS}{\boldsymbol{S}}
\providecommand{\bfT}{\boldsymbol{T}}
\providecommand{\bfU}{\boldsymbol{U}}
\providecommand{\bfV}{\boldsymbol{V}}
\providecommand{\bfW}{\boldsymbol{W}}
\providecommand{\bfX}{\boldsymbol{X}}
\providecommand{\bfY}{\boldsymbol{Y}}
\providecommand{\bfZ}{\boldsymbol{Z}}

\providecommand{\bfalpha}{\boldsymbol{\alpha}}
\providecommand{\bfbeta}{\boldsymbol{\beta}}
\providecommand{\bfgamma}{\boldsymbol{\gamma}}
\providecommand{\bfdelta}{\boldsymbol{\delta}}
\providecommand{\bfepsilon}{\boldsymbol{\epsilon}}
\providecommand{\bfvarepsilon}{\boldsymbol{\varepsilon}}
\providecommand{\bfzeta}{\boldsymbol{\zeta}}
\providecommand{\bfeta}{\boldsymbol{\eta}}
\providecommand{\bftheta}{\boldsymbol{\theta}}
\providecommand{\bfvartheta}{\boldsymbol{\vartheta}}
\providecommand{\bfiota}{\boldsymbol{\iota}}
\providecommand{\bfkappa}{\boldsymbol{\kappa}}
\providecommand{\bflambda}{\boldsymbol{\lambda}}
\providecommand{\bfmu}{\boldsymbol{\mu}}
\providecommand{\bfnu}{\boldsymbol{\nu}}
\providecommand{\bfxi}{\boldsymbol{\xi}}
\providecommand{\bfpi}{\boldsymbol{\pi}}
\providecommand{\bfvarpi}{\boldsymbol{\varpi}}
\providecommand{\bfrho}{\boldsymbol{\rho}}
\providecommand{\bfsigma}{\boldsymbol{\sigma}}
\providecommand{\bfvarsigma}{\boldsymbol{\varsigma}}
\providecommand{\bftau}{\boldsymbol{\tau}}
\providecommand{\bfupsilon}{\boldsymbol{\upsilon}}
\providecommand{\bfphi}{\boldsymbol{\phi}}
\providecommand{\bfvarphi}{\boldsymbol{\varphi}}
\providecommand{\bfchi}{\boldsymbol{\chi}}
\providecommand{\bfpsi}{\boldsymbol{\psi}}
\providecommand{\bfomega}{\boldsymbol{\omega}}
\providecommand{\bfGamma}{\boldsymbol{\Gamma}}
\providecommand{\bfDelta}{\boldsymbol{\Delta}}
\providecommand{\bfTheta}{\boldsymbol{\Theta}}
\providecommand{\bfLambda}{\boldsymbol{\Lambda}}
\providecommand{\bfXi}{\boldsymbol{\Xi}}
\providecommand{\bfPi}{\boldsymbol{\Pi}}
\providecommand{\bfSigma}{\boldsymbol{\Sigma}}
\providecommand{\bfUpsilon}{\boldsymbol{\Upsilon}}
\providecommand{\bfPhi}{\boldsymbol{\Phi}}
\providecommand{\bfPsi}{\boldsymbol{\Psi}}
\providecommand{\bfOmega}{\boldsymbol{\Omega}}

\providecommand{\bfOne}{\boldsymbol{1}}
\providecommand{\bfTwo}{\boldsymbol{2}}
\providecommand{\bfThree}{\boldsymbol{3}}
\providecommand{\bfFour}{\boldsymbol{4}}
\providecommand{\bfFive}{\boldsymbol{5}}
\providecommand{\bfSix}{\boldsymbol{6}}
\providecommand{\bfSeven}{\boldsymbol{7}}
\providecommand{\bfEight}{\boldsymbol{8}}
\providecommand{\bfNine}{\boldsymbol{9}}
\providecommand{\bfZero}{\boldsymbol{0}}

\providecommand{\bfinfty}{\boldsymbol{\infty}}

%Miscellaneous
\newcommand{\diff}{\,\mathrm{d}}
\newcommand{\quot}[1]{\enquote{#1}}
\newcommand{\setof}[1]{\widehat{#1}} %die Menge einer Algebra
\newcommand{\WLOG}{without loss of generality}
\DeclareMathOperator{\fst}{fst}
\DeclareMathOperator{\snd}{snd}
\DeclareMathOperator{\id}{id}

%Functions
\newcommand{\successor}[1]{s(#1)}
\newcommand{\predecessor}[1]{p(#1)}
\newcommand{\successors}[1]{S_{#1}}
\newcommand{\predecessors}[1]{P_{#1}}
\newcommand{\innerSet}[1]{\widehat{#1}}

%Operators for Relations
\newcommand{\equalDef}{\coloneqq}
\newcommand{\defEqual}{\eqqcolon}
\newcommand{\equivDef}{:\Leftrightarrow}

%Operators for Functions
\newcommand{\dProd}[2]{\left\langle #1, #2 \right\rangle}
\newcommand{\revdProd}[2]{\,\overleftrightarrow{\!\dProd{#1}{#2}\!}\,}
\newcommand{\revrevdProd}[2]{\,\overleftrightarrow{\overleftrightarrow{\!\dProd{#1}{#2}\!}}\,}
\newcommand{\dProdFunc}{\dProd{\cdot}{\cdot}}
\newcommand{\revdProdFunc}{\revdProd{\cdot}{\cdot}}
\newcommand{\revrevdProdFunc}{\revrevdProd{\cdot}{\cdot}}
\newcommand{\norm}[1]{\| #1 \|}
\newcommand{\logtwo}{\log_2}

%Operators for Sets
\newcommand{\powerset}[1]{\calP(#1)}
\newcommand{\card}[1]{|#1|}

\DeclareMathOperator{\Hom}{Hom}
\DeclareMathOperator{\Kern}{Kern}
\DeclareMathOperator{\Bild}{Bild}
%\DeclareMathOperator{\det}{det}
\DeclareMathOperator{\im}{im}
\DeclareMathOperator{\cod}{cod}
\DeclareMathOperator{\dom}{dom}
\DeclareMathOperator*{\argmax}{argmax}
\DeclareMathOperator*{\argmin}{argmin}
\DeclareMathOperator{\Span}{span}
\DeclareMathOperator{\Ker}{Ker}
\DeclareMathOperator{\ggT}{ggT}
\DeclareMathOperator{\diag}{diag}
\DeclareMathOperator{\intdiv}{div}

%Concrete Set Identifiers
\newcommand{\steadyOn}[1]{C(#1)}

\newcommand{\natOne}{\bbN^+}
\newcommand{\natZero}{\bbN_0}
\newcommand{\oneToN}[1]{\bbN^{(#1)}}
\newcommand{\integers}{\bbZ}
\newcommand{\rationals}{\bbQ}
\newcommand{\reals}{\bbR}
\newcommand{\complex}{\bbC}

\newcommand{\posReals}{\bbR^+}
\newcommand{\nonNegReals}{\bbR_0^+}
\newcommand{\negReals}{\bbR^-}
\newcommand{\nonPosReals}{\bbR_0^-}

\newcommand{\domain}[1]{\dom \left( #1 \right)}
\newcommand{\codomain}[1]{\cod \left( #1 \right)}
\newcommand{\image}[1]{\im \left( #1 \right)}

%Complexity Sets
%\newcommand{\compEq}[2][]{\Theta_{#1} \left( #2 \right)}
%\newcommand{\compLess}[2][]{o_{#1} \left( #2 \right)}
%\newcommand{\compLeq}[2][]{\calO_{#1} \left( #2 \right)}
%\newcommand{\compGreater}[2][]{\omega_{#1} \left( #2 \right)}
%\newcommand{\compGeq}[2][]{\Omega_{#1} \left( #2 \right)}
\newcommand{\compEq}[1]{\Theta \left( #1 \right)}
\newcommand{\compLess}[1]{o\left( #1 \right)}
\newcommand{\compLeq}[1]{\calO\left( #1 \right)}
\newcommand{\compGreater}[1]{\omega\left( #1 \right)}
\newcommand{\compGeq}[1]{\Omega\left( #1 \right)}
\newcommand{\subsEq}{\subseteq}
\newcommand{\subsNeq}{\subsetneq}

\newcommand{\assign}{\leftarrow}

\providecommand{\piValue}{3.1415926535897932384626433832795}

\newcommand{\mean}{\mathbb{E}}
\newcommand{\var}{\mathbb{V}}
\newcommand{\dd}{\text{d}}

\makeatletter
\renewcommand*\env@matrix[1][*\c@MaxMatrixCols c]{%
  \hskip -\arraycolsep
  \let\@ifnextchar\new@ifnextchar
  \array{#1}}
\makeatother

\newcommand{\ovl}{\overline}

%\DeclareCaptionSubType*[arabic]{figure}
%\captionsetup[subfigure]{labelformat=simple,labelsep=colon}

\newcommand{\lmax}{l_{\text{max}}}
\newcommand{\ntable}{\text{neighbors}}
\newcommand{\otable}{\text{orientations}}
\newcommand{\dir}{\mathrm{dir}}
\newcommand{\nDim}{\mathrm{nDim}}
\newcommand{\tableDepth}{\text{tableDepth}}
\newcommand{\orientation}{\text{orientation}}

% other variables
\newcommand{\etah}{\hat{\eta}}
\newcommand{\vh}{\hat{v}}
\newcommand{\bfvh}{\hat{\bfv}}
\newcommand{\Deltah}{\hat{\Delta}}

% -----------------------------------------------
% Notation
\newcommand{\model}{\mathcal{M}}
\newcommand{\uct}{\response^{\text{CT}}_\calL}
\newcommand{\upc}{\response^{\text{PC}}_\calP}
\newcommand{\upch}{\hat{\response}^{\text{PC}}_\calP}
\newcommand{\Fcal}{\mathcal{F}}

\newcommand{\response}{u}
\newcommand{\ctlevel}{\ell}
\newcommand{\level}{l}
\newcommand{\levelap}{l'}
\newcommand{\bflevel}{\boldsymbol{\level}}
\newcommand{\rv}{\xi}
\newcommand{\rvi}{\rv^{(i)}}
\newcommand{\rvs}{\boldsymbol{\xi}}
\newcommand{\rvsi}{\boldsymbol{\xi}^{(i)}}
\newcommand{\samplespace}{\Gamma}
\newcommand{\sample}{\gamma}
\newcommand{\stochasticspace}{\Xi}
\newcommand{\gridpointsn}{\stochasticspace^{(n)}}
\newcommand{\gridpointslevel}{\stochasticspace^{(\nlevel)}}
\newcommand{\gridpointslevelap}{\stochasticspace^{(n_{\levelap})}}
\newcommand{\gridpointsbflevel}{\stochasticspace^{(n_{\bflevel})}}
\newcommand{\basis}{\varphi}
\newcommand{\basislj}{\basis_{\level, j}}
\newcommand{\basisljk}{\basis^{(k)}_{\level, j}}

\newcommand{\transbasis}{\psi}
\newcommand{\transbasisk}{\transbasis^{(k)}}
\newcommand{\transbasisbfm}{\transbasis_{\bfm}}
\newcommand{\transbasislj}{\transbasis_{\level, j}}
\newcommand{\transbasisbfj}{\transbasis_{\bfj}}
\newcommand{\transbasisbfi}{\transbasis_{\bfi}}
\newcommand{\transbasisjkk}{\transbasis_{j_k}^{(k)}}
\newcommand{\transbasisikk}{\transbasis_{i_k}^{(k)}}

\newcommand{\dims}{d}

\newcommand{\nlevel}{n_\level}
\newcommand{\node}{\gp}
\newcommand{\nodei}{\gp_i}
\newcommand{\nodej}{\gp_j}

\newcommand{\xj}{x_j}
\newcommand{\xlj}{x_{l,j}}
\newcommand{\xilj}{x^{(i)}_{l,j}}
\newcommand{\xli}{x_{li}}
\newcommand{\xone}{x^{(1)}_{l_1 j_1}}
\newcommand{\xd}{x^{(d)}_{l_d j_d}}
\newcommand{\xseq}{(\xlj)_{j=1}^{\nl}}
\newcommand{\xbflj}{x_{\bflevel, \bfj}}
\newcommand{\vlj}{v_{l, j}}
\newcommand{\tvilj}{\tilde{v}^{(i)}_{l, j}}
\newcommand{\tvklj}{\tilde{v}^{(k)}_{l, j}}
\newcommand{\tvone}{\tilde{v}^{(1)}_{l_1 j_1}}
\newcommand{\tvd}{\tilde{v}^{(d)}_{l_d j_d}}
\newcommand{\vseq}{(\vlj)}

\newcommand{\tenC}{\bfT}
\newcommand{\Cbfm}{\tenC_{\bfm}}
\newcommand{\Cbfl}{\tenC_{\bflevel}}
\newcommand{\CCT}{\tenC^{\mathrm{CT}}_{\calL}}

\newcommand{\colvec}{\bfc}
\newcommand{\colveci}{c} % column vector indexed (i.e. it is only a scalar component)
\newcommand{\colten}{\bfC}
\newcommand{\colteni}{C} % column tensor indexed

\newcommand{\indexSet}{\calL}

\newcommand{\alg}{\calR}


\newcommand{\Ltwo}{$L^2$}
\newcommand{\Ltwoleja}{\Ltwo-Leja~}

\newcommand{\unit}{[0, 1]}

\newcommand{\deltaij}{\delta_{i, j}}
\newcommand{\gp}{\rv}
\newcommand{\gpli}{\gp_{l, i}}
\newcommand{\gplj}{\gp_{l, j}}
\newcommand{\gplik}{\gp_{l, i}^{(k)}}
\newcommand{\gpljk}{\gp_{l, j}^{(k)}}
\newcommand{\gplkik}{\gp_{l_k, i_k}}
\newcommand{\gplkjk}{\gp_{l_k, j_k}}
\newcommand{\gpone}{\gp^{(1)}_{\level_1, j_1}}
\newcommand{\gpd}{\gp^{(d)}_{\level_d, j_d}}
\newcommand{\gpbflj}{\gp_{\bflevel, \bfj}}
\newcommand{\gpbfli}{\gp_{\bflevel, \bfi}}

\newcommand{\pdf}{f}
\newcommand{\interpolant}{\tilde{\response}}
\newcommand{\Phit}{\tilde{\Phi}}

\newcommand{\Lnorm}[3]{\left\|#1\right\|_{L^{#2}(#3, f)}}
\newcommand{\Lqnorm}[1]{\Lnorm{#1}{q}{\stochasticspace}}
\newcommand{\Linftynorm}[1]{\Lnorm{#1}{\infty}{\stochasticspace}}
\newcommand{\Ltwonorm}[1]{\Lnorm{#1}{2}{\stochasticspace}}
\newcommand{\Ltwonormsquared}[1]{\Lnorm{#1}{2}{\stochasticspace}^2}
\newcommand{\Ltwonormsquaredinterval}[1]{\Lnorm{#1}{2}{[\node_i, \node_{i + 1}]}^2}
\newcommand{\Lonenorm}[1]{\Lnorm{#1}{1}{\stochasticspace}}

\newcommand{\LqnormSingle}[1]{\big\|#1\big\|_{L^q(\stochasticspace, f)}}

\newcommand{\abs}[1]{\left|#1\right|}

\newcommand{\rvseqj}{(\rv_j)_{j=1}^{n}}

\newcommand{\Vand}{\bfV}
\newcommand{\VandScalar}{V}
